%% LyX 2.3.7 created this file.  For more info, see http://www.lyx.org/.
%% Do not edit unless you really know what you are doing.
%\documentclass[english]{article}
\documentclass[UTF8]{ctexart}
\usepackage[T1]{fontenc}
%\usepackage[latin9]{inputenc}
\usepackage{geometry}
\geometry{verbose,tmargin=2.5cm,bmargin=2.5cm,lmargin=2.5cm,rmargin=2.5cm}
\setlength{\parskip}{\smallskipamount}
\setlength{\parindent}{0pt}
\usepackage{color}
\usepackage{babel}
\PassOptionsToPackage{normalem}{ulem}
\usepackage{ulem}
%\usepackage{CJKutf8}
\usepackage[unicode=true]{hyperref}

\def\TopicLabel{}

\ifdefined\TopicLabel
\def\A4M{\textcolor[rgb]{0.9,0,0.5}{\large{\tt[\#A4M]}}}
\def\M4A{\textcolor[rgb]{0,0,0.8}{\large{\tt[\#M4A]}}}
\def\EAI{\textcolor{red}{\large{\tt[\#EAI]}}}
\def\SUS{\textcolor[rgb]{0.3,0.8,0}{\large{\tt[\#SUS]}}}
\def\iDM{\textcolor[rgb]{0.5,0.5,1}{\large{\tt[i*]}}}
\def\ind#1{\hspace{-3em}#1\hspace{0.36em}}
\else
\def\A4M{}
\def\M4A{}
\def\EAI{}
\def\ind#1{}
\fi


\def\iA4M{\ind{\A4M}}
\def\iM4A{\ind{\M4A}}
\def\iEAI{\ind{\EAI}}

\makeatletter
\@ifundefined{date}{}{\date{}}
%%%%%%%%%%%%%%%%%%%%%%%%%%%%%% User specified LaTeX commands.
%\XeTeXlinebreaklocale "zh"
%\XeTeXlinebreakskip = 0pt plus 1pt

\makeatother

\begin{document}

\vspace{-5em}

\title{Zishuo Zhao 赵梓硕}
\maketitle

\vspace{-5em}

\begin{center}
zishuoz2@illinois.edu, wiku30@mit.edu
\par\end{center}

\vspace{-1.7em}

\begin{center}
ISE, University of Illinois Urbana-Champaign
\par\end{center}

\section{Background}
\begin{itemize}
\item \textbf{2023/09 - 2024/08} Visiting Student, Institute for Data Systems and Society (IDSS), \\
Massachusetts Institute of Technology\\
Advisor: David Simchi-Levi
\item \textbf{2021/01 - now }PhD student, Department of Industrial \& Enterprise
Systems Engineering, \\
University of Illinois Urbana-Champaign\\
Research Area: Operations Research\\
Advisor: Yuan Zhou\\
Expected Graduation: 2026 - 2028 (depending on my research progress)\\
Tentative Thesis Title: \emph{Reliable and Sustainable Platform Design for Digital Economy and AI} 
\item \textbf{2020/08 - 2021/01 }Research Assistant, Haihua Institute for
Frontier Information Technology\\
Research Area: Network Optimization\\
Advisor: Longbo Huang
\item \textbf{2016/05 - 2020/07} Undergraduate student, Institute
for Interdisciplinary Information Sciences (Yao Class), Tsinghua University\\
Research Area: Geometry-Based Visual SLAM\\
Advisor: Shi-Min Hu
\item \textbf{2015/08 - 2016/05} Undergraduate student, Department of Computer
Science and Technology, Tsinghua University 
\end{itemize}

\section{Research Statement}

I am a fourth-year PhD candidate in UIUC, majoring in operations research,
and currently doing research in mechanism design. In general, my research
interests span a wide scope related to incentive-aware design and optimization for
emerging applications in digital economy, including blockchain systems,
e-commerce, ridesharing platforms and so on. I am also interested in data-driven
mechanism design based on statistical/online learning, and topics in cryptography 
and distributed systems with applications in blockchain.

Particularly, I am recently interested in the exploration into a novel paradigm of \emph{incentive security}
on blockchain and AI platforms, which aims to combine game-theoretic and systematic/cryptographic methodologies
to prevent dishonest behavior of \emph{untrusted but rational} parties, for a general scope of reliability and social sustainability in AI and economy, including (particularly) the current concern and initiative of \textbf{AI Safety} for the upcoming AGI era.

On a high level, my \textcolor[rgb]{0.5,0.5,1}{(starry-eyed) dream} \iDM~lies in the exploration and design of \textbf{Sustainable Decentralized Trustworthy AI} platforms. Following are the topics I am actively working on:

\begin{itemize}
    \item \textbf{AI for mechanism design~\A4M}: e.g., using data-driven methods to 
    improve the performance of economic platforms.
    \item \textbf{Mechanism design for AI~\M4A}: e.g., using economic incentives to 
    reinforce the efficiency and security of AI systems, especially on the blockchain platform.
    \item \textbf{Sustainability in AI \& Economy~\SUS}: e.g., using interdisciplinary
    methodologies to foster long-term social welfare and environmental friendliness
    for new-era AI and economic platforms. 
\end{itemize}

Additionally, I was interested in geometry-based computer vision (Visual SLAM in particular) at
my undergraduate times. Although I no longer focus on that research field, I am
still happy to discuss about related topics, especially the recently fruitful field of
Embodied AI~\EAI, and open to possible future interdisciplinary collaboration after my graduation.


\section{Research}

%\setlength\parindent{24pt}
%\hangindent 24pt

\subsection{Publications and Preprints}

\textbf{Proof-of-Learning with Incentive Security} \iDM~\M4A~\SUS~\href{https://arxiv.org/abs/2404.09005}{[Link]}

\uline{Zishuo Zhao}, Zhixuan Fang, Xuechao Wang, Xi Chen, Yuan Zhou.

\emph{In Submission.}

Invited to \emph{INFORMS  Conference on Security (IConS'24)}.

\emph{ACM EC Workshop on Foundation Models and Game Theory (ACM EC FMGT'24).}

A preliminary version presented on \emph{INFORMS Annual Meeting 2023}. 

\medskip{}


\textbf{Bayesian Mechanism Design for Blockchain Transaction Fee Allocation}
%\footnote{A preliminary version of this research has the title ``Bayesian-Nash-Incentive-Compatible
%Mechanism for Blockchain Transaction Fee Allocation''} 
\M4A~\href{https://arxiv.org/abs/2209.13099}{[Link]}

Xi Chen{*}, David Simchi-Levi{*}, \uline{Zishuo Zhao}{*}, Yuan
Zhou{*}. (alphabetical order)

Major Revision in \textcolor{blue}{\emph{Operations Research}}

\textcolor{blue}{\href{https://ai-secure.github.io/DMLW2022/papers}{Best Paper Award}}, \emph{
NeurIPS Workshop on Decentralization and Trustworthy Machine Learning
in Web3 (NeurIPS DMLW'22)}.

\emph{Crypto Economics Security Conference (CESC 2022)}.

Invited to OR Talk by 运筹OR帷幄 (OR China) in 2024.

Invited to \emph{INFORMS Annual Meeting 2022}.

\medskip{}


\newpage

\textbf{Personalized Pricing with Group Fairness Constraint} \A4M~\SUS~\href{https://dl.acm.org/doi/10.1145/3593013.3594097}{[Link]}

Xin Chen{*}, Zexing Xu{*}, \uline{Zishuo Zhao}{*}, Yuan Zhou{*}.
(alphabetical order)

\emph{ACM Conference on Fairness, Accountability, and Transparency
(ACM FAccT 2023)}

\medskip{}



\textbf{Dynamic Car Dispatching and Pricing: Revenue and Fairness
for Ridesharing Platforms } \A4M~\SUS~\href{https://arxiv.org/abs/2207.06318}{[Link]}

\uline{Zishuo Zhao}, Xi Chen, Xuefeng Zhang, Yuan Zhou

\emph{International Joint Conference on Artificial Intelligence (IJCAI
2022)}, \textbf{Long Oral (3.75\%)}. 

Invited to \emph{INFORMS Annual Meeting 2021}.

\medskip{}



\textbf{ClusterSLAM: A SLAM Backend for Simultaneous Rigid Body Clustering
and Motion Estimation} \EAI~\href{https://link.springer.com/article/10.1007/s41095-020-0195-3}{[Link]}

Jiahui Huang, Sheng Yang, \uline{Zishuo Zhao}, Yukun Lai, Shi-Min
Hu.

\emph{Computational Visual Media (CVM), Volume 7, pages 87–101 (2021).}

\emph{International Conference on Computer Vision (ICCV 2019). }


\subsection{Working Projects}


\textbf{(Peer Prediction for) Decentralized Verification Game} \iDM~\M4A 

\uline{Zishuo Zhao}, Xi Chen, Yuan Zhou

\emph{(an interest-driven project; a preliminary version: \href{https://arxiv.org/abs/2406.01794}{[Link]})}

\medskip{}

\textbf{Incentive-Aware Dynamic Auction for Budgeted Bidders} \A4M

David Simchi-Levi{*}, \uline{Zishuo Zhao}{*}, Yuan Zhou{*}. (alphabetical order)








\section{Awards}

\subsection{Academic}
\begin{itemize}
\item \textbf{Best Paper Award}, NeurIPS Workshop on Decentralization and
Trustworthy Machine Learning in Web3, 2022
\item Crypto Economics Security Conference (CESC) Travel Award, 2022
\item UIUC Graduate College Conference Presentation Award, 2021
\item 12th place in 2nd THUCTF Cybersecurity Contest in Tsinghua
University, 2020
\item 12th place in 24th Artificial Intelligence Programming Contest in
Tsinghua University, 2020

\item Xuetang Scholarship in Tsinghua University, 2016-2020
\item 15th place in 20th Artificial Intelligence Programming Contest in
Tsinghua University, 2016. 
\item Second Prize in Chinese Mathematical Olympiad (CMO), 2014
\item First Prize in National Olympiad in Informatics in Provinces (NOIP),
2013
\end{itemize}

\subsection{Arts}

\begin{itemize}
    \item Finalist Award in the ``一盏茶时 (\textit{IIIS Tea Time})'' photography exhibition for the 12th Anniversary of IIIS, Tsinghua University, 2023. 
    \item Scholarship for Arts Excellence in Tsinghua University, 2018.
    \item Third Prize in the Art Festival of No.1 Middle School affiliated to CCNU, 2013. 
\end{itemize}

\section{Academic Activities}
\begin{itemize}
\item Invited to OR Talk by 运筹OR帷幄 (OR China) in 2024.
\item Conference reviewing: ESA 2024.
\item Facilitator in section Revenue \& Pricing, INFORMS Annual Meeting, 2021.
\item Invited to the Alumni Forum for the 10th Anniversary of IIIS,
Tsinghua University in 2021.
\end{itemize}

\section{Miscellaneous}

\subsection{Languages}
\begin{itemize}
\item Mandarin Chinese (native)
\item English (fluent)
\item Classical Chinese (writing as a hobby)
\end{itemize}

\subsection{Extracurricular Activities}
\begin{itemize}
\item I have an amateur interest in Capture-The-Flag (CTF) cybersecurity competitions, and developed two Reverse challenges in the TQLCTF 2022.
\item I have a wide scope of hobbies in arts and aerobic sports, especially in vocal music, photography, piano, table tennis and orienteering. 
Particularly, my photographs ``梦影(\textit{Dreamland Impression})'' and ``$2\pi i$'' won the Finalist Award in the photography exhibition for the 12th Anniversary of IIIS, Tsinghua University in 2023.
\end{itemize}

\subsection{Coding Skills}
\begin{itemize}
\item Mainly using C++, MATLAB and Mathematica, also with command of Python,
Java, PHP.
\item With some knowledge in ROS, Verilog HDL and assembly language.
\item Open to learn new programming languages when in need.
\end{itemize}
\begin{flushright}
\emph{(Updated on Aug 1, 2024)}
\par\end{flushright}
\end{document}
